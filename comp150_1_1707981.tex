\documentclass{scrartcl}

\usepackage[hidelinks]{hyperref}
\usepackage[none]{hyphenat}
\usepackage{setspace}
%\doublespace Plz no, I need to proofread this

\title{How can agile philosophy maximise learning in an academic game development project?}
\subtitle{COMP150 - Agile Development Practice}
\date{2017-11-16}
\author{1707981}

\begin{document}
\maketitle
\pagenumbering{arabic}


\abstract{Today, game development-specific courses teach beyond the traditional specialist skills -- such as art and programming -- into territories that are typically learned within the work environment. These include communication and team management in project-oriented lessons. However, with a small team of students, a project and a scrum workflow, many issues arise on a personal basis when learning and adapting specialist skills whilst concurrently producing a project. With some minor changes the agile philosophy may be adapted to maximise the learning of these skills in a positive and motivated environment.}
% Abstract

\section{Introduction}
A key principle in the Agile philosophy is iteration \cite{agile}. The iterative process of continually improving according to previous mistakes virtually matches the definition of learning. However, agile philosophy declares that working software is the primary measurement of progress \cite{manifesto}, and is focused on producing an ideal product in a short timeframe -- contrast to learning how to do it with good practice. This can raise several issues in a traditional group environment, particularly where skill gaps appear \cite{group2003}.
% Agile, what it is and why it isn't education

This study aims to discover methods to improve the learning of production skills during the production process itself, taking place in a hypothetical academic agile game production environment incorporating the scrum framework. While this is a deliberately academic context, significant overlap with game industry practice is expected; as learning, reflection and teamworking are widely acknowledged as vital skills \cite{collaboration, devstudy}.
% What we're exploring today, kids! In a "hypothetical" environment

% ==================== ~200 WORDS~ ======================

\section{Why scrum?}
One of scrum's characteristics is the scrum board, wherein team members choose the tasks they plan to do \cite{scrum}. Choosing one's own tasks is believed to boost motivation by promoting autonomy \cite{motivation}, leading to deeper engagement with the team.
% Autonomy and learning

A project owner may be tempted to choose an agile workflow not to learn, but to improve the actual game. A finished game will often be very different to its initial design, due to its dependence on unpredictable factors such as look and feel \cite{collaboration}. As a result, most game developers adopt a similarly iterative workflow as an essential part of the early process \cite{iteration, olddays, devstudy}.
% Why choose scrum

However, game development companies as a majority don't use an agile workflow. While evidence suggests that most game industry workflows are ``agile'' philosophically, (at least in the early stages \cite{olddays}); many create their own personalised workflows, and others use no particular workflow at all \cite{devstudy}. This calls to question whether agile development is truly appropriate for games, or whether adjustments should be made, although its benefits to team autonomy can't be ignored.

% ===================== ~300 WORDS~ ====================

\section{Learning pitfalls}
A paper by L. J. Barker and Kathy Garvin-Doxes \cite{group2003} highlighted a startling concern that student group projects, despite their practicality, can in fact inhibit learning. With the pressure to deliver an equal work standard, being in a team of more experienced developers can lessen an inexperienced students' own desire to contribute.

Psychology and motivation are at the core of these issues. One influential study \cite{motivation} observed that peers who were encouraged to learn by an extrinsic goal, such as money, learned less than those who were encouraged with an intrinsic goal, such as understanding themselves better. The game is an extrinsic goal, which could direct focus to the game's quality at the expense of learning. This puts pressure on inexperienced team members, and can introduce the problem of `unrealistic scope' often shared by the industry \cite{problems}, wherein a project becomes too demanding for a team to reasonably deliver.
% Reason behind the problems, slight game industry mention

These pitfalls call to question how agile philosophy could accommodate them. It is, however, arguable that agile already does this through the `interaction' clause \cite{manifesto}, especially compared to, for example, a more traditional \textit{waterfall} workflow. In a \textit{waterfall} workflow, teams are typically fixed to a single plan for an extended period with minimal process review \cite{waterfall}. Much unlike agile, this would lack individual autonomy, would be focused on an extrinsic goal (the plan), and would not accommodate self-reflection. Meanwhile, the strong single-track focus on producing the work would introduce pressure as noted in  \cite{group2003}.
% Agile vs waterfall

\section{Improving}
\subsection{Intrinsic goals}
Boosting peer discussion of these key issues could improve the situation. To recover motivation, awareness of the intrinsic goals could be gently raised by adding a 'what I want to learn today' clause to the scrum stand-ups. This may overlap with 'what I plan to do today', and so may seem redundant. However, simply raising one's intrinsic learning goals could itself reproduce the positive results shown in the study \cite{motivation}; and perhaps more importantly, give more knowledgeable team members an opportunity to help.

\subsection{Reflection}
The same could go for a 'what I learned yesterday' clause: It is well-documented that groups who reflect on their learning process, rather than just the task they achieved, are typically more successful in their endeavours \cite{effectivegroups, learnreflection}. Reflection is already incorporated into most agile workflows for similarly educational reasons \cite{agile, scrum}, except they have a greater focus on the product. In this environment, educational focus should be added. When communicated clearly, reflection may also spark discussion among peers about the learning topics in specific, and bring to attention those who understand it and those who wish to learn.
% Promoting motivation, and the benefits, etc

\subsection{Peer learning}
Peer learning is actively practiced in industry, where openness to learning \cite{devstudy} and willingness to help others \cite{collaboration} are vital elements to a team member's personality. Encouraging the application of additional individual-oriented support, such as pair programming between peers of ideally varied skill gaps, would help to emphasise the essential goal of sharing knowledge. It may also aid troubleshooting in situations where an unknown problem is hindering production \cite{collaboration, motivation}.
% Peer learning and how to improve the skill gaps

In fact, \cite{group2005} also found working in pairs to be a considerable solution to the skill gap. While it may be viewed as a burden, the act of knowledge sharing itself is known to have a positive effect not only on the learner, but the teacher as well \cite{activepassive, motivation}. 

\section{Conclusion}
The agile workflow highly accomodates learning in a group environment: improving peers' motivation, ability to learn in the workplace, communication skills, all of which are believed to be vital to a game developer's employability. However, this is impacted by individual personalities; and low motivation levels, attitude and skill gaps remain key obstacles to learning. Scrum mitigates these by promoting autonomy and communication, but could further mitigate them by implementing a learning-oriented tailoring of the framework; supporting peer learning, greater intrinsic awareness, and discussion of learning goals and achievements.
% Key points, weaknesses and areas of potential improvement

Outstanding questions include how personality conflicts could be resolved, whether highly motivated individuals would better learn through group work or personal study, and/or whether further group workflow adaptations can accommodate all of these issues. However, the importance of collaboration and learning is well-known in industry; and beyond the educational institution, these employability skills are essential. It is thus arguable that agile philosophy by its nature is a good starting point in supporting the development of the developer.

% Oustanding questions
% ========= 250 WORDS ========

\bibliography{references} 
\bibliographystyle{ieeetr}

\end{document}