\documentclass{scrartcl}

\usepackage[hidelinks]{hyperref}
\usepackage[none]{hyphenat}
\usepackage{setspace}
%\doublespace Plz no, I need to proofread this

\title{How can agile philosophy maximise skill learning in an small game development teams?}
\subtitle{COMP150 - Agile Development Practice}
\date{\today}
\author{1707981}

\begin{document}
\maketitle
\pagenumbering{arabic}

\abstract{Game development is more than a process of creating games. It is a process of learning, collaboration and discovery of what is required to create a unique and interesting experience for the player. The effects of learning from a product's development are inspirational to a team, but it also calls to question how much they can learn to produce while producing. An exploration of learning tendencies reveals Agile development's striking ability to inspire a team's skill learning throughout the collaborative development process, and proposes some potential techniques to augment this positive effect.}
% Abstract

\section{Introduction}
Agile philosophy promotes, among many other things, interaction, iteration and individuals \cite{manifesto, agile}

Some companies create a strong emphasis on learning during development \cite{collaboration}, and it is a highly desired skill to the industry in general, with 48\% of employers placing it in their Top 3 requirements in a recent global survey \cite{devstudy}.

\section{Agile}
Take the simplest waterfall approach, characterised by rough sequential process including analysis of requirements, design, coding, then testing \cite{waterfall, olddays}. Contrast to Agile, which sacrifices the predictive plan in favour of a dynamic framework that accommodates regular change and communication. \cite{manifesto, agile} It is naturally arguable that Agile is most fitting to the game industry: a game concept is rarely fully established at the beginning \cite{collaboration}, making it challenging to write a linear plan that can accommodate the unpredictable results of playtesting. As such, most game companies take iterative approaches, some extremely variable with additive, reductive forms of iteration \cite{iteration} applied in industry.

The waterfall approach, however, does have its uses. In larger teams, regular meetings and interactions with other members becomes hard due to the large numbers of people. 

\section{The trends}
Game industry workflows are surprisingly varied, with many choosing unstructured workflows \cite{devstudy}. This trend suggests that setting their own workflow is, in a sense, another part of the game team's creative process; though most of them are still Agile by philosophy in that they often share similar views on iteration \cite{iteration} and collaboration \cite{collaboration}. However, a high proportion still use the traditional waterfall model \cite{devstudy,olddays}. That is not to say these developers do not learn from their working practice--in fact writing postmortems is a common method of reflecting on previous game development experiences \cite{problems}, opening opportunities to improve. Both agile and non-agile teams can exploit this.

Whether or not agile philosophy is superior for game development is beyond the scope of this study; but the educational power of reflection \cite{reflection} and iteration \cite{iteration, collaboration} are well-proven, and it is arguable that regular reflection as described in the Agile principles \cite{principles} and cemented in frameworks such as Scrum \cite{scrum} is a more effective learning tool owing to its higher regularity.

\section{Tensions}
A paper by L. J. Barker and Kathy Garvin-Doxes from 2003\cite{group2003} highlighted the threat of skill gaps in team projects. A negative impact on inexperienced contributors' motivation was observed when they were surrounded by more skilled students. In the studio, this is equivalent to junior and senior positions. Agile already promotes 

\section{A helping hand}

\section{Conclusion}

\bibliography{references} 
\bibliographystyle{ieeetr}

\end{document}