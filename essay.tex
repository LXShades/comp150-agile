\documentclass{scrartcl}

\usepackage[hidelinks]{hyperref}
\usepackage[none]{hyphenat}
\usepackage{setspace}
%\doublespace Plz no, I need to proofread this

\title{How can agile philosophy maximise learning in an academic game development project?}
\subtitle{COMP150 - Agile Development Practice}
\date{2017-11-16}
\author{1707981}

\begin{document}
\maketitle
\pagenumbering{arabic}

\abstract{Game development courses promise to equip students with the skills to develop videogames, now teaching beyond the specialist skills traditionally taught by institutions--such as art and programming--into territories that are typically learned within the work environment, including communication and team management. In practice, with a small team of students, a project and a scrum workflow, many issues arise on a personal basis when learning and adapting specialist skills whilst concurrently producing a product. However, with some minor changes the agile philosophy could be adapted to maximise the learning of these skills in a positive and motivated environment.}
% This abstract is 666 characters long so you can trust it

\section{Introduction}
The key principle in education is learning, whilst a key principle in the Agile philosophy is iteration \cite{agile}. Iteration is described as a continous improvement based on reflection of previous results \cite{iteration}. Such reflection is hugely beneficial to learning new skills in a personal context \cite{learnreflection} [CITEMORE?]; however, iteration in the context of agile development is focused on producing an ideal product in an ideal timeframe, stating that working software is the primary measure of progress, \cite{manifesto} contrast to learning how to do it with good practice.  This can raise several issues in a traditional group environment, particularly where skill gaps arise \cite{group2005}.
% Agile, what it is and why it isn't education

This study is interested in discovering possible solutions to incorporating the learning of production skills into production itself. Its goal is to explore potential methods and techniques to improve learning, motivation and productivity in a hypothetical academic agile game production environment incorporating the scrum framework--this will be the context for the remainder of the study; however, while this context is academic, we expect to find a lot of overlap wherein the same learning techniques are experienced in the game industry itself.
% What we're exploring today, kids! In a "hypothetical" environment

% ==================== ~200 WORDS~ ======================

\section{Workflow suitability}
The scrum agile framework is characterised by the scrum board, wherein team members choose their own tasks \cite{scrum}. Choosing one's own tasks has proven benefits in both education and industry by promoting autonomy, which is a vital source of motivation, \cite{motivation} and making the group's individual tasks explicit enhances learning. \cite{group2005}
% Autonomy and learning

... iteration ...
% WIP

In addition, it is highly suited to the game industry itself. Iteration by nature is highly respected in the game industry, often considered vital \cite{iteration}[CITEMORE?], owing to the fun factor, playtesting, and reviews continuously impacting the direction the project takes. Some even consider the iterative process to be enjoyable \cite{iteration}; this further improves motivation.[CITE.........]
% Iteration in industry

% ===================== ~300 WORDS~ ====================

\section{Addressing issues}
However, despite its promises and the many benefits it offers to group work, agile fails to address some key pitfalls. Granted, it is arguable that many of the pitfalls are common far beyond game development itself--and that they would be further aggravated by the adoption of a more traditional waterfall workflow[CITE], wherein teams are typically fixed to a single plan for an extended period with little room for reflection. This approach lacks autonomy reflection and is focused on an extrinsic goal--the plan, which has a negative impact \cite{motivation}, \cite{learningreflection} on learning.
% Agile vs waterfall closing (WIP)

A paper by L. J. Barker and Kathy Garvin-Doxes from 2003 highlighted a startling concern that group projects, despite their practicality, can in fact inhibit personal learning. In particular, being in a team of more experienced developers can lessen some inexperienced students' own desire to contribute. In some cases, the burden of delivering an equal standard of work can sour motivation, and students drop out entirely \cite{group2003}. Furthermore, in efforts to deliver high quality work, many students prefer to work in their "comfort zone", rather than expanding into areas they have yet to learn.\cite{group2003} \cite{group2005} The lack of learning is contrary to the interests of an educational institution, and in fact the game industry itself\cite{collaboration}, \cite{devstudy}.
% About the negatives of group work

These problems are centred not strictly on the workflow, but psychology and motivation. A study detailed in \cite{motivation} observed that students who were encouraged to learn by an extrinsic goal, such as money, learned less than those who were encouraged with an intrinsic goal, such as understanding themselves better. The game is an extrinsic goal, which may risk diluting the focus of learning and teaching the skills, and redirecting it to the game's quality. This puts pressure on students, especially the inexperienced, and a common issue in the game industry manifests--unrealistic scope \cite{problems}, wherein a project is too demanding for a team to reasonably deliver on time.
% Reason behind the problems, slight game industry mention

To combat these issues, perhaps the biggest focus should be on improving motivation. In the Scrum environment, awareness of the intrinsic goals, as described in \cite{motivation}, could be applied by adding a 'what did I learn yesterday' part to the stand-ups. Some may consider it redundant, as it may overlap with 'what did I create yesterday'; however, simply raising the intrinsic goal itself could produce same positive results as in the study \cite{motivation}. It is in fact well-documented that groups who reflect on their learning process, rather than just the task, are typically more successful in their endeavours \cite{effectivegroups}, \cite{learningreflection}. This extra awareness may also spark voluntary discussion among peers about the newly learned material in specific, and bring to attention others who are also interested in learning it--providing the benefits of peer learning and autonomy at the same time. This could be further augmented by considering a 'what do I want to learn today' clause, which would help to briefly shift focus from the project to the individual, giving more knowledgeable team members to an opportunity to autonomously volunteer to help.
% How to improve motivation and success

In fact peer learning should be encouraged team-wide. Despite the extra burden, the act of active teaching is shown to have as positive an effect on the teacher as the learner \cite{activepassive}, and is actively practiced in industry \cite{devstudy}[CHECK?], where openness to learning \cite{devstudy} and willingness to help others \cite{collaboration} are vital elements to a team member's personality. [CITEMORE?]. This will particularly benefit the lesser skilled individuals, mitigating some of the key issues above. Encouraging the application of additional individual-oriented support, such as pair programming between peers of both similar or varied skill gaps, would help to emphasise the essential goal of sharing knowledge, and aid troubleshooting in situations where an unknown problem is hindering development \cite{collaboration} [CHECK] \cite{motivation} \cite{devstudy}.
% Peer learning and how to improve the skill gaps

\section{A comparison to industry}
Learning doesn't stop at university [CITE], learning is ongoing etc [CITE]. A survey conducted between 2014-2015 discovered that of the top required skills for hiring a new employee, 'ability to learn while working' \cite{devstudy} was the most prevalent, with 48\% of game developers citing this in their Top 3 priorities.
% Industry on learning, this could probably be moved

% Game bringing team members together. Cite more sources for game industry transfer?

Identifying technical mistakes is a challenge to those who are unaware, and is mostly achieved purely through trial and error \cite{capstone}. 
% About inexperienced groups are inexperienced, what's new?

From an organisational perspective, arranging groups into a smaller ranges of past experience is shown to be beneficial from a motivational standpoint. A study from 2009 \cite{peerreview} amusingly discovered that while students prefer to learn from the experienced, in a blind experiment they found reviews from lesser experienced team members to be more useful.\cite{peerreview} However, conversely, heterogenous teams--teams varied in skill--are often seen among the most successful, but this depends on the team's ability to communicate. \cite{group2003} Therefore a focus should be put on communication, something which is already learned during practical application of the agile philosophy. Being in a heterogenous groupalso helps when wntering the game industry as a junior[REVISE AND CITE].
% Interesting but inconclusive viewpoints on skill gaps

\section{Conclusion}
The agile workflow highly accomodates learning in a group environment with several strengths; improving peers' motivation, ability to learn in the workplace, communication skills, all of which are proven to be vital skills to a game developer's employability. However, this is impacted by vital areas of individual personality; and low motivation levels, attitude and skill gaps can create tensions in the team. Agile philosophy mitigates these through the promotion of autonomy and strong communication of work progress, but they could, dependent on personalities, be further mitigated by implementing a learning-oriented tailoring of the Scrum framework, promoting more peer learning, more awareness and discussion of learning goals and achievements, and deliberate application of pair programming.
% Key points, weaknesses and areas of potential improvement

As a relatively new teaching area, there is much to explore in creating an ideal learning environment for aspiring game developers. Outstanding questions particularly include whether there are better, more creative ways to further benefit the learning environment. There are many possible answers and no single solution, and it is worth recommending that agile student team workflows continue to be improved over time through iteration and self-reflection for some cool meta agile shoite.
% Oustanding questions

\bibliography{references} 
\bibliographystyle{ieeetr}

\end{document}