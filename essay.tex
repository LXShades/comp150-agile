\documentclass{scrartcl}

\usepackage[hidelinks]{hyperref}
\usepackage[none]{hyphenat}
\usepackage{setspace}
%\doublespace Plz no, I need to proofread this

\title{How can agile philosophy maximise learning in an academic game development project?}
\subtitle{COMP150 - Agile Development Practice}
\date{2017-11-16}
\author{1707981}

\begin{document}
\maketitle
\pagenumbering{arabic}

\abstract{Game development courses promise to equip students with the skills to develop videogames, now teaching beyond the specialist skills traditionally taught by institutions -- such as art and programming -- into territories that are typically learned within the work environment, including communication and team management. In practice, with a small team of students, a project and a scrum workflow, many issues arise on a personal basis when learning and adapting specialist skills whilst concurrently producing a project. However, with some minor changes the agile philosophy may be adapted to maximise the learning of these skills in a positive and motivated environment.}
% Abstract

\section{Introduction}
A key principle in education is learning, whilst a key principle in the Agile philosophy is iteration \cite{agile}, described as a continous improvement based on reflection of previous results \cite{iteration}. However, agile philosophy states that working software is the primary measurement of progress \cite{manifesto}, and appears to be focused on producing an ideal product in an ideally short timeframe -- contrast to learning how to do it with good practice. This can raise several issues in a traditional group environment, particularly where skill gaps appear \cite{group2003}.
% Agile, what it is and why it isn't education

This study aims to discover potential methods to improve the learning of production skills during the production process itself. It will attempt to improve learning and productivity in a hypothetical academic agile game production environment incorporating the scrum framework. While this is a deliberately academic context, significant overlap with industry practices is expected, as learning, reflection and teamworking are widely acknowledged as vital skills in the game industry (\cite{collaboration}, \cite{devstudy}).
% What we're exploring today, kids! In a "hypothetical" environment

% ==================== ~200 WORDS~ ======================

\section{Why scrum?}
The hypothetical small team adopts the scrum agile framework. This framework is characterised by the scrum board, wherein team members choose their own tasks \cite{scrum}. Choosing one's own tasks has proven benefits in both education and industry by promoting autonomy, which is a vital source of motivation \cite{motivation}, and explicitly expressing the group's individual tasks enhances learning. \cite{group2005}
% Autonomy and learning

An educational environment may be tempted to choose an agile workflow not just for its learning benefits, but for its relevance in industry. A finished game will often be very different to its initial design, due to its dependence on unpredictable qualities such as look and feel \cite{collaboration}; as a result, most companies in industry adopt an iterative workflow as a vital part of the process \cite{iteration}, \cite{olddays}, \cite{devstudy}. Evidence suggests that most game industry workflows are agile by nature \cite{olddays}; although interestingly, game development companies often create their own personalised workflows, not self-proclaimed as ``agile'', and many use no particular workflow at all \cite{devstudy}.
% WIP

% ===================== ~300 WORDS~ ====================

\section{Pitfalls}
Despite its many benefits, a paper by L. J. Barker and Kathy Garvin-Doxes from 2003 highlighted a startling concern that group projects, despite their practicality, can in fact inhibit learning. In particular, being in a team of more experienced developers can lessen some inexperienced students' own desire to contribute. In some cases, the burden of delivering an equal standard of work can sour motivation, and students drop out entirely \cite{group2003}. Furthermore, in efforts to deliver high quality work, many students prefer to work in their "comfort zone", rather than expanding into areas they have yet to learn \cite{group2003}, \cite{group2005}. The lack of learning is contrary to the interests of an educational institution, and in the game industry itself \cite{collaboration}, \cite{devstudy}.
% About the negatives of group work

These problems are centred not strictly on the agile workflow, but psychology and motivation. One influential study \cite{motivation} observed that students who were encouraged to learn by an extrinsic goal, such as money, learned less than those who were encouraged with an intrinsic goal, such as understanding themselves better. The game is an extrinsic goal, which could direct focus to the game's present quality at the expense of learning. This puts pressure on inexperienced team members, and can raise the issue of 'unrealistic scope' already well-known in the industry itself \cite{problems}, wherein a project becomes too demanding for a team to reasonably deliver. [Cite postmortems?]
% Reason behind the problems, slight game industry mention

Many of these pitfalls are common far beyond game development itself, and it is arguable that they would be further aggravated by the adoption of a more traditional waterfall workflow \cite{waterfall} wherein teams are typically fixed to a single plan for an extended period, with minimal in-production reviews. This approach lacks individual autonomy, is focused on an extrinsic goal (the plan), and doesn't promote reflection -- three things which could be vital to learning.
% Agile vs waterfall

\section{Going upwards}
To combat these issues, a focus could be placed on promoting spirit and motivation. In the Scrum environment, awareness of the intrinsic goals could be gently raised by adding a 'what did I learn yesterday' clause to the stand-ups. This may perhaps overlap with 'what I did yesterday'; however, simply raising the intrinsic goal of learning itself could reproduce the positive results shown in the study \cite{motivation}. It is, in fact, well-documented that groups who reflect on their learning process, rather than just the task, are typically more successful in their endeavours \cite{effectivegroups}, \cite{learnreflection}; and reflection is already incorporated into most agile workflows \cite{agile}, \cite{scrum} to accommodate the product's quality.
% Promoting motivation, and the benefits, etc

Reflection on learning may also spark discussion among peers about the learning subject in specific, and bring to attention other members also interested in learning it -- providing the benefits (\cite{collaboration}, \cite{motivation}, \cite{group2005}) of peer learning in a fully autonomous way. This could be further augmented with a 'what do I want to learn today' clause, which would help an individual express their learning goals, and give more knowledgeable team members an opportunity to help. 
% How to improve motivation and success

In fact peer learning should be encouraged team-wide, owing to its several benefits. It is actively practiced in industry, where openness to learning \cite{devstudy} and willingness to help others \cite{collaboration} are vital elements to a team member's personality. Furthermore, the act of knowledge sharing itself is known to have a positive effect on both the teacher and learner \cite{activepassive} \cite{motivation}. Encouraging the application of additional individual-oriented support, such as pair programming between peers of both similar or varied skill gaps, would help to emphasise the essential goal of sharing knowledge, and aid troubleshooting in situations where an unknown problem is hindering development \cite{collaboration} \cite{motivation} \cite{devstudy}, particularly for the inexperienced.
% Peer learning and how to improve the skill gaps

\section{Conclusion}
The agile workflow highly accomodates learning in a group environment: improving peers' motivation, ability to learn in the workplace, communication skills, all of which are proven as vital skills to a game developer's employability. However, this is impacted by vital areas of individual personality, and low motivation levels, attitude and skill gaps remain key obstacles to learning. Scrum mitigates these with promotion of autonomy and strong communication of work progress, but they could be further mitigated by implementing a learning-oriented tailoring of the framework, adding focus to peer learning (such as pair programming), greater intrinsic awareness, and discussion of learning goals and achievements.
% Key points, weaknesses and areas of potential improvement

As a relatively new teaching topic, there is much to explore in creating an ideal learning environment for aspiring game developers. While agile workflows have positive benefits to motivation and therefore learning, it remains inconclusive comparitively whether highly motivated individuals will learn better specialist practices through group work or personal study, and/or whether this can be accommodated with further group workflow adaptations. There is doubtless no single answer; making it logical that agile student team workflows should continue to be improved over time through iteration and self-reflection like some cool meta agile shite.

% Oustanding questions
% ========= 250 WORDS ========

\section{Recycle bin}
\section{A comparison to industry}
As hinted in the earlier findings, a survey conducted between 2014-2015 discovered that of the top required skills for hiring a new employee, 'ability to learn while working' \cite{devstudy} was the most prevalent, with 48\% of game developers citing this in their Top 3 priorities.
% Industry on learning, this could probably be moved

% Game bringing team members together. Cite more sources for game industry transfer?

Identifying technical mistakes is a challenge to those who are unaware, and is mostly achieved purely through trial and error \cite{capstone}. 
% About inexperienced groups are inexperienced, what's new?

From an organisational perspective, arranging groups into a smaller ranges of past experience is shown to be beneficial from a motivational standpoint. A study from 2009 \cite{peerreview} amusingly discovered that while students prefer to learn from the experienced, in a blind experiment they found reviews from lesser experienced team members to be more useful.\cite{peerreview} However, conversely, heterogenous teams--teams varied in skill--are often seen among the most successful, but this depends on the team's ability to communicate. \cite{group2003} Therefore a focus should be put on communication, something which is already learned during practical application of the agile philosophy. Being in a heterogenous groupalso helps when wntering the game industry as a junior[REVISE AND CITE].
% Interesting but inconclusive viewpoints on skill gaps

\bibliography{references} 
\bibliographystyle{ieeetr}

\end{document}